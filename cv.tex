\documentclass{cv}

\usepackage[left=0.75in,top=0.6in,right=0.75in,bottom=0.6in]{geometry}
\usepackage{amsmath,amssymb,amsthm}
\usepackage{enumitem}
\usepackage[hidelinks]{hyperref}
\newcommand{\tab}[1]{\hspace{.2667\textwidth}\rlap{#1}}
\newcommand{\itab}[1]{\hspace{0em}\rlap{#1}}
\name{Psi Vesely} % Your name
\address{psi@ucsd.edu} % Your phone number and email

%----------------------------------------------------------------------------
%----------------------------------------------------------------------------
\begin{document}

%----------------------------------------------------------------------------
%----------------------------------------------------------------------------
\begin{rSection}{Education}

\textbf{UCSD} \hfill \emph{`20---Present}  \\
PhD in Cryptography \hfill \emph{San Diego, CA} \\
{\small Thesis proposal to be made late '21/early '22---likely post-quantum zero-knowledge proofs.}

\textbf{University College London} \hfill \emph{`18---`19}  \\
Master in Information Security \emph{with distiction} \hfill \emph{London, UK} \\
Thesis: \emph{On the Design of Polynomial Commitment Schemes} \\
{\small We introduce two new extractable polynomial commitments (PCs). One with \emph{succinct} proofs, enforceable degree bounds, and an \emph{updatable} SRS. The other \emph{transparent} with constant-sized commitment, square root opening time and CRS size, and log-time verification.}

\textbf{Hampshire College} \hfill \emph{`10---`15} \\
Bachelor of Mathematics, minor in Computer Science \hfill \emph{Amherst, MA} \\
Thesis 1: \emph{On McEliece-Type Cryptosystems as Post-Quantum Standards} \\
{\small An analysis of McEliece-Type cryptosystems as possible post-quantum standards, focusing on developments towards an IND-CCA2 variant with sufficiently small key size for embedded devices.} \\
Thesis 2: \emph{Evolving a Cryptographic Compression Function} \\
{\small We introduce a new cryptographic compression function evolved using the PushGP genetic programming environment and a novel fitness heuristic.}

\end{rSection}

%----------------------------------------------------------------------------
%----------------------------------------------------------------------------
\begin{rSection}{Publications}

\textbf{Proofs for Inner Pairing Products and Applications} \hfill \emph{Oct. `19} \\
\emph{In submission, ASIACRYPT `21} \\
\url{https://eprint.iacr.org/2019/1177.pdf} \\
{\small We present a generalized inner product argument (GIPA), then show how
to achieve a log-time verifier for several pairing-based languages. We apply
these arguments to build polynomial commitments that improve on the opening
proof time and CRS size of KZG, provide the first concretely efficient protocol for aggregating Groth16 proofs without recursion, and construct a low-memory SNARK improving on Bitansky et al. [STOC '13].}

\textbf{Marlin: Preprocessing zkSNARKs with Universal and Updatable SRS} \hfill \emph{Sept. `19} \\
\emph{EUROCRYPT `20} \\
\url{https://eprint.iacr.org/2019/1047.pdf} \\
{\small We present a methodology to construct preprocessing zkSNARKs where the structured reference string (SRS) is universal and updatable. Improving on the state-of-the-art in this setting, we use our methodology to obtain a preprocessing zkSNARK where the SRS has linear size and arguments have constant size.}

\end{rSection}

%----------------------------------------------------------------------------
%----------------------------------------------------------------------------

\begin{rSection}{Work experience}


\textbf{cLabs} \hfill \emph{June `19---Present} \\
\emph{Research Scientist} \hfill \emph{San Francisco, CA} \\
{\small Designed an ultralight client for the Celo blockchain using SNARKS and circuit-friendly primitives including a novel aggregeate multisignature
composite hash function. Co-author of the technical paper.}

\textbf{University of California, Berkeley} \hfill \emph{Oct. `19---Oct. `20} \\
\emph{Research Assistant} \hfill \emph{Berkeley, CA} \\
{\small With professor Alessandro Chiesa, researching topics in zero-knowledge proofs.}

\textbf{Glyff} \hfill \emph{Summer `18} \\
\emph{Research Scientist} \hfill \emph{London, UK} \\
{\small Designed a privacy-preserving smart contract platform based on Ethereum. Co-authored the white paper and audited an initial implementation.}

\textbf{Freelance} \hfill \emph{Sept. `17---May `18} \\
\emph{Security Engineer} \hfill \emph{Mexico City, MX} \\
{\small Information security and cryptography related development, consulting, and training. With a focus on helping non-profits, clients included Data Cívica and Human Rights Data Analysis Group.}

\textbf{Freedom of the Press Foundation} \hfill \emph{Sept. `15---Aug. `17} \\
\emph{Security Engineer} \hfill \emph{San Francisco, CA} \\
\url{https://github.com/freedomofpress/securedrop} \\
{\small Jointly developed the SecureDrop open-source whistleblower submission platform. Stringent security requirements and a multi-machine, multi-OS architecture demanded wide-breadth domain knowledge including cryptographic, network, OS, and application-level security expertise.} \\
\url{https://github.com/freedomofpress/fingerprint-securedrop} \\
{\small Implemented a machine learning system to evaluate website fingerprinting attacks and defenses for Tor onion services. Led Tor developer conference sessions on the topic and worked closely with academic researchers.}

\end{rSection}

%----------------------------------------------------------------------------
%----------------------------------------------------------------------------
\begin{rSection}{Invited talks}

\textbf{Scaling Bitcoin} at Tel Aviv University \hfill \emph{Sept. `19} \\
\emph{The Celo Ultralight Client} \hfill \emph{Tel Aviv, ISR}

\textbf{COSIC Group} at KU Leuven \hfill \emph{Mar. `17} \\
\emph{Fingerprinting SecureDrop} \hfill \emph{Leuven, BE}

\end{rSection}

%----------------------------------------------------------------------------
%----------------------------------------------------------------------------
\begin{rSection}{Teaching experience}

\textbf{Hampshire College Quantitative Resource Center} \hfill \emph{Sept. `12---May `15} \\
\emph{Manager} \hfill \emph{Amherst, MA} \\
{\small Tutored primarily mathematics and computer science. Managed the hiring process, budgeting, and staff schedule, leading to record attendance levels.}

\textbf{Linear Algebra} \hfill \emph{Fall `14} \\
\emph{Teaching Assistant} \hfill \emph{Hampshire College} \\
{\small Guest taught a lecture, held weekly office hours, and marked coursework with Prof. Sarah Hews.}

\end{rSection}

%----------------------------------------------------------------------------
%----------------------------------------------------------------------------
\begin{rSection}{Academic service}

\textbf{Financial Cryptography} \hfill \emph{2020} \\
\emph{Subreviewer} \\

\end{rSection}

%----------------------------------------------------------------------------
%----------------------------------------------------------------------------

\begin{rSection}{Software Projects}

\textbf{ripp} \hfill \emph{`20} \\
\emph{Co-Author} \hfill \url{https://github.com/arkworks-rs/ripp} \\
{\small An implementation of several arguments from the ``Proofs for Inner Pairing Products and Applications'' paper.}

\textbf{marlin} \hfill \emph{`19} \\
\emph{Co-Author} \hfill \url{https://github.com/arkworks-rs/marlin} \\
{\small An implementation of the Marlin zkSNARK.}

\textbf{winternitz} \hfill \emph{`18} \\
\emph{Author} \hfill \url{https://github.com/nvesely/winternitz} \\
{\small The first standalone implementation of the post-quantum WOTS-T one-time signature scheme.}

\textbf{Rusty Secrets} \hfill \emph{`18} \\
\emph{Co-Author} \hfill \url{https://github.com/SpinResearch/RustySecrets} \\
{\small A Rust implementation of Shamir's Secret Sharing Scheme that provides authentication of shares. Used in the Sunder application.}

\textbf{SodiumOxide} \hfill \emph{`18} \\
\emph{Maintainer \& Contributor} \hfill \url{https://github.com/sodiumoxide/sodiumoxide} \\
{\small A Rust interface to the C++ libsodium cryptography library that seeks to utilize Rust's featureset to improve on the usability and safety of the library.}

\textbf{libalpaca} \hfill \emph{`17} \\
\emph{Co-Author} \hfill \url{https://github.com/camelids/libalpaca}\\
{\small A library that implements ALPaCa, an application-layer defense against website fingerprinting.}

\end{rSection}


%----------------------------------------------------------------------------
%----------------------------------------------------------------------------
\end{document}
%----------------------------------------------------------------------------
%----------------------------------------------------------------------------

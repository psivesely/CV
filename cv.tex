\documentclass{cv}

\usepackage[left=0.75in,top=0.6in,right=0.75in,bottom=0.6in]{geometry}
\usepackage{amsmath,amssymb,amsthm}
\usepackage{enumitem}
\usepackage[hidelinks]{hyperref}
\newcommand{\tab}[1]{\hspace{.2667\textwidth}\rlap{#1}}
\newcommand{\itab}[1]{\hspace{0em}\rlap{#1}}
\name{Noah `Psi' Vesely} % Your name
\address{psi@berkeley.edu} % Your phone number and email

%----------------------------------------------------------------------------
%----------------------------------------------------------------------------
\begin{document}

%----------------------------------------------------------------------------
%----------------------------------------------------------------------------
\begin{rSection}{Education}

\textbf{University College London} \hfill \emph{`18---`19}  \\
Master in Information Security \emph{with distiction} \hfill \emph{London, UK} \\
Thesis: \emph{On the Design of Polynomial Commitment Schemes} \\
{\small We introduce two new extractable polynomial commitments (PCs), one \emph{succinct} and with \emph{updatable} SRS, and another \emph{transparent} and practically efficient with logarithmic proofs and commitments, and sublinear verification.}

\textbf{Hampshire College} \hfill \emph{`10---`15} \\
Bachelor of Mathematics, minor in Computer Science \hfill \emph{Amherst, MA, USA} \\
Thesis 1: \emph{On McEliece-Type Cryptosystems as Post-Quantum Standards} \\
{\small An analysis of McEliece-Type cryptosystems as a possible post-quantum PKC standard. Review of literature focusing on developments towards an IND-CCA2 variant with sufficiently small key size for embedded devices.} \\
Thesis 2: \emph{Evolving a Cryptographic Compression Function} \\
{\small We introduce a new cryptographic compression function evolved using the PushGP genetic programming environment and a novel fitness heuristic.}

\end{rSection}

%----------------------------------------------------------------------------
%----------------------------------------------------------------------------
\begin{rSection}{Publications}

\textbf{Aggregatable Signatures from an Inner Pairing Product Argument} \hfill \emph{Oct. `19} \\
\url{https://eprint.iacr.org/2019/1177.pdf} \\
{\small We present a new public-coin setup argument for membership in pairing-based languages and use it to build a new logarithmic-size, BLS-based aggregate signature that can be verified much faster than previous results.}

\textbf{Marlin: Preprocessing zkSNARKs with Universal and Updatable SRS} \hfill \emph{Sept. `19} \\
\emph{In submission, EUROCRYPT `20} \\
\url{https://eprint.iacr.org/2019/1047.pdf} \\
{\small We present a methodology to construct preprocessing zkSNARKs where the structured reference string (SRS) is universal and updatable. Improving on the state-of-the-art in this setting, we use our methodology to obtain a preprocessing zkSNARK where the SRS has linear size and arguments have constant size.}

\end{rSection}

%----------------------------------------------------------------------------
%----------------------------------------------------------------------------

\begin{rSection}{Work experience}

\textbf{University of California, Berkeley} \hfill \emph{Oct. `19---Present} \\
\emph{Research Assistant} \hfill \emph{Berkeley, CA} \\
{\small With professor Alessandro Chiesa, researching topics in zero-knowledge proofs and theoretical cryptography.}

\textbf{C Labs} \hfill \emph{June `19---Present} \\
\emph{Research Scientist} \hfill \emph{San Francisco, USA} \\
{\small Designed an ultra-light client sync protocol for the Celo BFT-style blockchain network utilizing zkSNARKS, proof-carrying data, and SNARK-friendly primitives including a novel composite hash function. Writing the technical paper.}

\textbf{Glyff} \hfill \emph{Summer `18} \\
\emph{Research Scientist} \hfill \emph{London, UK} \\
{\small Designed a privacy-preserving smart contract platform based on Ethereum. Co-authored the white paper and audited an initial implementation.}

\textbf{Freelance} \hfill \emph{Sept. `17---May `18} \\
\emph{Security Engineer} \hfill \emph{Mexico City, MX} \\
{\small Information security and cryptography related development, consulting, and training. With a focus on helping non-profits, clients included Data Cívica and Human Rights Data Analysis Group.}

\textbf{Freedom of the Press Foundation} \hfill \emph{Sept. `15---Aug. `17} \\
\emph{Security Engineer} \hfill \emph{San Francisco, USA} \\
\url{https://github.com/freedomofpress/securedrop} \\
{\small Developed the SecureDrop open-source whistleblower submission platform. Stringent security requirements and a multi-machine, multi-OS architecture demanded wide-breadth domain knowledge including cryptographic, network, OS, and application-level security expertise.} \\
\url{https://github.com/freedomofpress/fingerprint-securedrop} \\
{\small Designed and implemented a machine learning system to evaluate website fingerprinting attacks and defenses for Tor onion services. Led Tor developer conference sessions on the topic.}

\end{rSection}

%----------------------------------------------------------------------------
%----------------------------------------------------------------------------
\begin{rSection}{Invited talks}

\textbf{Scaling Bitcoin} at Tel Aviv University \hfill \emph{Sept. `19} \\
\emph{The Celo Ultralight Client} \hfill \emph{Tel Aviv, ISR}

\textbf{COSIC Group} at KU Leuven \hfill \emph{Mar. `17} \\
\emph{Fingerprinting SecureDrop} \hfill \emph{Leuven, BE}

\end{rSection}

%----------------------------------------------------------------------------
%----------------------------------------------------------------------------
\begin{rSection}{Teaching experience}

\textbf{Hampshire College Quantitative Resource Center} \hfill \emph{Sept. `12---May `15} \\
\emph{Manager} \hfill \emph{Amherst, MA, USA} \\
{\small Tutored primarily mathematics and computer science. Managed the hiring process, budgeting, and staff schedule, leading to record attendance levels.}

\textbf{Linear Algebra} \hfill \emph{Fall `15} \\
\emph{Teaching Assistant} \hfill \emph{Hampshire College}

\end{rSection}

%----------------------------------------------------------------------------
%----------------------------------------------------------------------------
\begin{rSection}{Academic service}

\textbf{Financial Cryptography} \hfill \emph{2020} \\
\emph{Subreviewer} \\

\end{rSection}

%----------------------------------------------------------------------------
%----------------------------------------------------------------------------

\begin{rSection}{Other Software Projects}

\textbf{marlin} \hfill \emph{`19} \\
\emph{Co-Author} \hfill \url{https://github.com/scipr-lab/marlin} \\
{\small An implementation of the Marlin zkSNARK.}

\textbf{winternitz} \hfill \emph{`18} \\
\emph{Author} \hfill \url{https://github.com/nvesely/winternitz} \\
{\small The first standalone implementation of the post-quantum WOTS-T one-time signature scheme.}

\textbf{SodiumOxide} \hfill \emph{`18} \\
\emph{Maintainer \& Contributor} \hfill \url{https://github.com/sodiumoxide/sodiumoxide} \\
{\small A Rust interface to the C++ libsodium cryptography library that seeks to utilize Rust's featureset to improve on the usability and safety of the library.}

\textbf{Rusty Secrets} \hfill \emph{`18} \\
\emph{Co-Author} \hfill \url{https://github.com/SpinResearch/RustySecrets} \\
{\small A Rust implementation of Shamir's Secret Sharing Scheme that provides authentication of shares. Used in the Sunder application.}

\textbf{libalpaca} \hfill \emph{`17} \\
\emph{Co-Author} \hfill \url{https://github.com/camelids/libalpaca}\\
{\small A library that implements ALPaCa, an application-layer defense against website fingerprinting.}

\end{rSection}


%----------------------------------------------------------------------------
%----------------------------------------------------------------------------
\end{document}
%----------------------------------------------------------------------------
%----------------------------------------------------------------------------